\documentclass[11pt,a4paper]{article}

%----------------------------------------------------
\usepackage{graphicx,amsmath,amssymb}
\usepackage{fancyhdr}
%\usepackage{subfig}
\usepackage{paralist}
\usepackage{caption}
\captionsetup{font=small,labelsep= period}
\graphicspath{{../../figures/}}
\usepackage[a4paper,top=3cm,bottom=3.5cm,left=2cm,right=2cm]{geometry}
%\usepackage{hyperref}
\usepackage{epsfig}

\title{Some notes about nsPipeflow}
\author
{
  Jose M. Lopez, Liang Shi, Markus Rampp, Bj\"orn Hof \& Marc Avila\\
  }
  \date{\today}

%--------------------------------------------------------
\begin{document}

\maketitle

%\newpage


\noindent The code \verb+nsPipeflow+ merges the numerical formulation
of Openpipeflow [1], an open source code to simulate pipe flow,
with the hybrid parallel strategy of nsCouette [2], a HPC
code for DNS of Taylor-Couette flow, creating a highly scalable solver
to simulate pipe flow at large values of the Reynolds number. For details
about the equations and numerical methods please read the documentation
of Openpipeflow (http://www.openpipeflow.org/index.php?title=Manual).
From a user perspective \verb+nsPipeflow+ works in the same way as \verb+nsCouette+,
and so the reader is encouraged to read in detail its documentation
(https://gitlab.rzg.mpg.de/mjr/nscouette/) before using \verb+nsPipeflow+. Here,
we will only briefly discuss some I/O parameters or operations that have been
modified with respect to \verb+nsCouette+.

\section{Input file: input\_nsPipeflow}

The input file \verb+input_nsPipeflow+ is the only file that
the user needs to modify to perform simulations with
\verb+nsPipeflow+. The only difference with respect to the
input file in \verb+nsCouette+ (\verb+input_nsCouette+)
are the physical parameters, which are naturally those specific
for pipe flow. There are two physical parameters in
\verb+input_nsPipeflow+. The Reynolds number (\verb+Re+), based on the centreline
velocity of the laminar flow and the pipe radius, and a logical
variable \verb+const_flux+ that controls whether the fluid motion along
the pipe is driven by a constant pressure gradient (\verb+const_flux=F+)
or a constant mass flux (\verb+const_flux=T+).

\section{Initial perturbation}

When the initilization option \verb+restart=0+ is chosen,
the simulation is started from the base flow (Hagen-Poiseuille),
which is perturbed during the initialization phase. The
disturbances added to the base flow are implemented in the subroutine
pulse\_init contained in mod\_InOut.f90. By default,
\verb+nsPipeflow+ implements a pair of streamwise localized
rolls ($v=A(g+rg') cos(\theta) e^{-wsin^2(\pi z/L_z)}$ and
$u=A g sin(\theta) e^{-wsin^2(\pi z/L_z)}$, where $u$ and $v$ are the
radial and azimuthal velocity components, $L_z$ is the pipe axial length,
$r$,$\theta$ and $z$ are the three spatial directions in cylindrical
coordinates, $w$ fixes the axial length of the perturbation,
$g=(1-r^2)^2$,  and $A$ is the amplitude of the
disturbance. This perturbation is known to produce puffs at the
transitional regime and turbulence at higher Reynolds numbers [3].
Nevertheless, the user may need to modify $A$ and $w$
depending on the values of $Re$ and $L_z$ considered in the simulations.


\section{Outputs}

\verb+nsPipeflow+ produces the same output files as \verb+nsCouette+,
with the exception of \verb+torque+ and \verb+vel_mid+. Instead,
a file \verb+friction+ is generated that contains time series
for the excess of friction (if a constant flux is set) or bulk velocity
(if a constant pressure gradient is chosen), the centreline velocity,
the friction velocity and the friction coefficient. In addition,
the azimuthally and axially averaged velocity profile is also written
every time that a checkpoint is saved. 



\section*{References}
\begin{enumerate}
\item Willis, A. P. The Openpipeflow Navier–Stokes solver. SoftwareX 6, 124–127 (2017)
\item Liang Shi, Markus Rampp, Bj\"orn Hof and Marc Avila, A hybrid MPI-OpenMP parallel implementation for
  pseudospectral simulations with application to Taylor-Couette
  flow. Computers \& Fluids 106 (2015) 1-11. (arXiv:1311.2481)
\item Fernando Mellibowski, Alvaro Meseguer, Tobias M. Schneider and Bruno Eckhardt,
  Transition in Localized Pipe Flow Turbulence (2009) PRL
  103,054502 (2009)
\end{enumerate}




\end{document}
